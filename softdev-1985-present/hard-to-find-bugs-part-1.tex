The interesting bugs tend to stay in your memory for a very long time. My first
memorable one dates back to 1993. It was my first "real" job after college. I was
working for a telecom company specializing in military communications. Since I
was the new grunt on the team, my first assignment was the bug that no one
wanted to deal with. The defect was low priority from the customer's viewpoint,
but they voiced mild irritation every few months. It was occasionally annoying,
but not a show-stopper.
\par
The system in question was ancient, even by 1993 standards. It was a mil spec
voice and data communications console. The guts of the beast consisted of
off-the-shelf PDP-11 hardware and various custom communication boards, complete
with the requisite gobs of firmware. Yes, the core really was a PDP-11 removed
from its traditional chassis. The console had many features, and one of them was
providing a basic secure phone. The bug I was investigating was not in the
high-end features, but in the simple phone keypad. For one out of every 100 or
so key presses, the tone (DTMF) for a key would stick until the user hung up and
dialed again.
\par
After some code spelunking, I discovered that the keypad firmware sent commands
via a DUART to a tone generator. The generator accepted simple commands to start
and stop tones (big surprise). Solving this was going to be easy, I thought. The
problem was likely a simple logic issue where the stop tone command was not
always sent when it should be. After reviewing the circuit schematics for the
board, I was pleased to discover that the designers had confxligured one channel
of the DUART as a debug port, with a small header for the crusty, yet trusty
2-3-7 (TX, RX, and GND of RS-232). This configuration allowed even the village
idiot to make a cable that could easily be connected to the serial port of any
PC, workstation, or VTXXX terminal.
\par
Believe it or not, there was a culture back then such that software folks had to
battle with hardware designers to make small concessions such as an RS-232 port
for firmware debugging. The argument was always something like "That will add 50
cents to the cost of every board! Why can't you software people get it together
before we go into production?" When confronted with this in meetings, I always
wanted to pipe up with "Well, let's talk about all the white wires that needed
to be applied to boards X,Y and Z, after they were put into production". I never
actually stated my true thoughts on the matter, since I accepted (capitulated)
that it would be in vain. I was a subordinate software grunt in a world of
hardware hubris.
\par
Before I forget, I should mention one particularly horrible hardware blunder.
Several hundred boards made it to production with some data lines reversed. I
don't recall exactly what communications circuitry was involved (maybe an RS422
link), but the end result was that we had to work around the problem in software
by mirroring every byte before writing to the transmitting device, and mirror
again when reading. A simple lookup table did the trick efficiently since we
were only dealing with 8 bits at a time, but from an engineering and common
sense perspective, it sure did not feel right. Economically however, it was the
right thing to do. I recall that there were many other less severe hardware
blunders my team had to work around. There was certainly an attitude back then
that hardware was never at fault. Perhaps this culture still exists. I have not
worked in the embedded world since 1995, therefore I am not a contemporary
authority on the subject. Further discussion on this topic is warranted, but for
now, back to the bug.
\par
My first course of action was to create a debug build of the firmware that
contained critical logging to an RS-232 port. At the time, doing this seemed
perfectly natural. In hindsight, it seems primitive. The board I was debugging
had no file system. There was not any infrastructure to log via a network or
hardware bus to some central data store. The RS-232 port was my only logging
mechanism. I needed to verify that the firmware was not sending the stop tone
when the annoying problem manifested itself. When I went to insert the necessary
logging, I quickly realized that I was the first to need the debug port. The
released firmware did not configure the debug port at all. At the time I smiled
when this became apparent. Back then I loved creating C structs and unions to
program hardware registers. The inane intricacy had some primitive appeal. This
probably explains why I suffered from an infatuation with C++ many years later.
After writing a bit of C code, a fresh compile, and an EPROM burn, I now had the
ability to log debug messages via the serial port.
\par
I did not have any reliable mechanism to repeat the error. I had to bang on the
keypad many times until a tone would stick. Usually 80 or so key presses would
trigger the bug. In retrospect, I probably could have saved some time by
creating debug code in the firmware to simulate keypad events. Hindsight is
always 20/20. What I did do however was log all the commands being written to
the tone generator. I expected that logging would show when the tone did not
terminate properly, the stop tone command was not sent due to a race condition,
or logic bug. I would then proceed to track down the race or logic problem. To
my surprise, the VT220 connected to my freshly minted debug port showed that the
stop tone command was indeed being written to DUART transmit register when the
bug manifested itself. Unfortunately, I was not well versed in all the nuances
of the DUART, so my next assumption was that it was likely some type of hardware
problem. I reserved an expensive logic probe (I think it was a techtronix) from
the equipment room and began setting it up to analyze the data bus of the DUART.
I expected to observe that the data lines would not contain the stop tone
command bits when the problem occurred. After running a logic trace on the
DUART's data bus, it became apparent that the stop command was indeed being
written to the data bus correctly.
\par
It had become very obvious that this might take a little longer to figure out.
By this time, I had about 3 or 4 days invested in the problem. I now knew that
the software was sending the stop tone command correctly and the data lines were
propagating this command to the DUART, and yet the tone was not not ceasing. My
next course of action was to use the probe to look at the bytes being sent to
the tone generator. Perhaps the generator was the problem? Again using the logic
probe, I monitored the transmit bytes via the TXD pin of the DUART. Ah ha!
Finally progress! The stop command was never making it to the tone generator. I
was happy, but I still had no solution. What would cause the DUART not to
transmit bytes written to its transmit register? I had demonstrated via the
software logging and the logic probe that the appropriate command was being
written to the DUART. Could the DUART be defective? Highly unlikely. Millions of
them were in production. There must be some internal state of the DUART such
that writing a byte to the transmit register was not guaranteed to actually be
transmitted. I began to pour over the DUART data sheet looking for a reason for
this behavior. It did not take long to find a plausible explanation. I quickly
found a likely candidate:

Bit 2 of the Status Register TxRDY - When set, it indicates that the
transmit-holding register (the one waiting to be transmitted) is ready to be
loaded with a new character.

Perhaps the firmware was not checking this bit before writing to the transmit
register? I then took another look at the transmitting code, and sure enough, it
was not checking this bit! A quick mod, and the problem was finally solved!

\par
Someone much more experienced with a DUART would probably have immediately
thought of checking for the proper usage of the TxRDY bit. However, this was a
great learning experience for me. I was new at the company, and it accelerated
my knowledge of my employer's processes such as source control usage, ROM image
archiving, and how to use a logic probe to debug software. It was also very
educational for me to look at circuit schematics alongside code. I did not work
in the embedded world for long, but the 'cool' factor of using a hardware probe
to debug software will always be a fond memory.